%
% Copyright (C) 2013 Yamauchi Hitoshi
%
%defparentheses	�� ��
%defswap	|
%defdelete	/
%defcomment	;
%defescape
%deforder	older-first
%defversion	0.9.5-current
%%% Local Variables:
%%% coding: euc-japan
%%% End:

%--- begin for platexj
% \documentclass[a4paper]{article}
%--- end for platex
%--- begin for latex
\documentclass[12pt]{scrartcl}
\usepackage{CJK}
%--- end for latex

%\usepackage{multicol}
\usepackage{amsmath}
\usepackage{a4wide}
\usepackage{graphicx}
\usepackage{cite}
\usepackage{color}
\usepackage{blkarray}
%\usepackage{authblk}
%\usepackage{footmisc}
% \usepackage{hyperref} % Unfortunately, this does not work with CJK
% \usepackage[ngerman, american]{babel}
\usepackage[american]{babel}
\graphicspath{{./fig/}}
% \href{URL}{text}

\sloppy

%%--- begin for latex
%% latex �ˤ� kanjiskip ���ʤ���kanjiskip �� platex �γ�ĥ
\newlength{\kanjiskip}
\setlength{\kanjiskip}{0pt}
%%--- end for latex
%%----------------------------------------------------------------------
%% Ruby  Hitoshi Yamauchi
%%
\def\ruby#1#2{%
    \leavevmode
    \setbox0=\hbox{#1}\setbox1=\hbox{\tiny#2}%
    \ifdim\wd0>\wd1 \dimen0=\wd0 \else \dimen0=\wd1 \fi
    \hbox{\kanjiskip=\fill
        \vbox{\hbox to \dimen0{\tiny \hfil#2\hfil}%
        \nointerlineskip
        \hbox to \dimen0{\hfil#1\hfil}}}}
%%----------------------------------------------------------------------
% math vector
\def\vec#1{\mbox{\boldmath {$#1$}} }

\begin{document}

%--- begin for latex
\begin{CJK*}[dnp]{JIS}{min}
%--- end for latex

\title{Graphfetcher user guide}
\date{2013-1-26(Sat)}

% 2013-1-30(Wed)

\author{Hitoshi Yamauchi}

%\twocolumn[]
\maketitle
% \tableofcontents

%\begin{multicols}{2}

%\begin{abstract}
\paragraph{Abstract}
This is a user guide of graphfetcher tools. graphfetcher tools are for
analyzing a Wiki's graph structure and computing the rank of the web
pages by eigenanalysis (also known as Google PageRank). I use this tool
for my computational literature research project. Though, it can be used
in any area that is written in Wikipedia, e.g., ranking of the musician,
ranking of politician, and so on.
%\end{abstract}

\section{Introduction}

One of my friends who studies literature asked me how I would analyze
the relationships between authors. For instance, ``How can we measure
the influence of Shakespeare in English literature compared to other
writers?''

This question inspired me to develop these tools. The detail of the
story was published as an article ``Authors in a Markov matrix: Which
author do people find most inspiring?''~\cite{bib:wikianalysis}.

You can find the following tools here.
\begin{itemize}
 \item \verb|Link_Vector_Extractor|: Generate the author vector from the
       data.
 \item \verb|Graph_Extractor|: Generate the adjacency matrix from: the
       data and the author vector.
 \item \verb|Page_Rank|: Compute PageRank.
 \item \verb|Remapper|: Re-map the author vector according to the
       PageRank result.
\end{itemize}
These tools are developed on a notebook computer, CPU: Intel(R) Core(TM)
DUO CPU P8400, 2 Cores, OS: 64bit Linux 3.2.0.32, Kubuntu 12.04. The
programming environment is: Python 2.7.3, Beautiful Soup 4.0.2, matlab
R2006a, octave 3.2.4.

\section{A step by step example}

% I visited Pisa to join my friend's wedding. I told him this project and
% then he requested me to show the Italian writer's rank in the sense of
% Wikipedia graph structure.

Here I will show you an example using the list of Italian writers.

\subsection{Download the writer's pages}

I first need download the writer's Wiki pages. I found a list of Italian
writer Wekipage,
\verb|http://en.wikipedia.org/wiki/|\verb|List_of_Italian_writers|. I
use this as the root page for the download.

\begin{verbatim}
% cd graphfetcher
% mkdir -p data/italian_writer
% cd data/italian_writer
% cp ../english_writer/ wget_download_en_wiki_en_writer.sh \
  wget_download_en_wiki_de_writer.sh
\end{verbatim}
I edit the \verb|wget_download_en_wiki_de_writer.sh| to change the list
and run the program. After download the pages, please create an output
directory, \verb|wiki_out| at the downloaded directory. For instance, if
you are at \verb|graphfetcher/data/italian_writer|, and English wiki
pages are downloaded, you have \texttt{en.wikipedia.org} directory,
\begin{verbatim}
% mkdir -p en.wikipedia.org/wiki_out
  # path: graphfetcher/data/italian_writer/en.wikipedia.org/wiki_out
\end{verbatim}


\subsection{Getting the author vector}

First we need an author vector. To extract this vector, we use the
\texttt{vectorextractor} tool. One easy way is reuse one of the
\verb|test_linkvectorextractor_*.py| file.
\begin{verbatim}
% cd ../vectorextractor
% cp test_linkvectorextractor_en_en_0.py \
  test_linkvectorextractor_italian_en_0.py
\end{verbatim}

These programs are also unit tests. There are baseline file comparison
test in the test methods: \verb|test_linkvectorextractor_ascii|,
\verb|test_linkvectorextractor_utf8|. For the first time run, you should
disable these comparison test, otherwise you will get test failed
errors.

\subsection{Getting the adjacency matrix}

From the author vector and all the wiki pages, we can analyze the link
structure for each authors. We generate an adjacency matrix as a
result. We use \texttt{graphextractor} to get the matrix. Same as the
\texttt{vectorextractor}, it is easy to copy the one of
\texttt{graphextractor} test and modify it.
\begin{verbatim}
% cd ../graphextractor
% cp test_graphextractor_en_en_0.py \
  test_graphextractor_italy_en_0.py
\end{verbatim}

The default options are silent. You can find the option settings in the
\verb|test_graphextractor_italy_en_0.py| file, search \verb|opt_dict|
definition.  If you want to know what is processed, change the following
parameters:
\begin{itemize}
 \item \verb|'log_level'|\texttt{: 3}
 \item \verb|'is_print_connectivity'|\texttt{: True}
\end{itemize}
This settings reports the analysis result of each page's link structure.
This program is also a unit test program. Therefore, the result
comparison is performed. For the first time run, there is no baseline,
so you will see the error in that case.

This tool can also output a dot graph file. But if the vector size is
more than 100, it may not help to see the structure of the graph.

There are two kinds of graph output in the current code. matlab's sparse
matrix format and my own format. If you need another format to support,
you should develop your exporting code.


\subsection{Compute the Eigenvector}

\textbf{Note: This needs matlab software.} This is mathematical analysis
part. I use matlab. Please see the m-file
\verb|test_pagerank_italian_en_0.m|. You should update the author vector
file name and adjacent matrix function in the following \texttt{madj}
line in the m-file.
\begin{itemize}
 \item \texttt{madj =} \verb|italian_en_writer_adj_mat();|
\end{itemize}

There is also a tool to visualize the sparse matrix. See
\verb|test_show_adjacency_mat_0.m|.


\subsection{Remap the result}

My program use the index of the author vector as identifier. Because the
PageRank algorithm has the sink link removal. The vector size may change
during the processing. I used the author name as the identifier in my
first implementation, but, it turned out the matlab's utf-8 support was
not sufficient for me. A utf-8 string becomes a number array, so you can
not see the character. Therefore, I switched the identifier to the index
of the vector. After PageRank is computed, the PageRank values are
associated with the indices. The remapper re-maps the indices to the
author names.

Please see \verb|test_remapper_0.py|.

\section{Tips}

\begin{itemize}
 \item wget version should be >= 1.13.4. (1.12 has a bug of utf-8
       filename conversion. This has been fixed 1.13.4.)
 \item If you see UnicodeEncodeError (e.g. UnicodeEncodeError: 'ascii'
       codec can't encode character u'\textbackslash{}xf2' in position
       *: ordinal not in range(*)), set utf-8 environment. For instance,
       environment variable, \verb|export LC_ALL=en_US.utf-8| in bash.
       You usually should be able to read the utf-8 file name in your
       terminal.
\end{itemize}


\section{Conclusion}

I tested in this document with Italian writers on English wikipedia.


% \(M_c: A \mapsto C\)

\bibliographystyle{plain}
\bibliography{bib}

%\end{multicols}
%--- begin for latex
\end{CJK*}
%--- end for latex

\end{document}
